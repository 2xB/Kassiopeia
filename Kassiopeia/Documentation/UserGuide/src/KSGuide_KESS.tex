% KESS documentation for the Kassiopeia Guide



\hypertarget{KESSMain}{}\section{KESS -\/ KATRIN Electron Scattering in Silicon}\label{KESSMain}
\begin{DoxyVersion}{Version}
1.4 \hyperlink{_k_e_s_s_changelog}{ChangeLog} 
\end{DoxyVersion}
\begin{DoxyAuthor}{Author}
Pascal Renschler (\href{mailto:pascal.renschler@ik.fzk.de}{\tt pascal.renschler@ik.fzk.de}) 
\end{DoxyAuthor}
\begin{DoxyDate}{Date}
2008-\/2010 
\end{DoxyDate}
\begin{DoxyAttention}{Attention}
The source of this documentation can be found in: /Modules/Transport/include/KESSRunManager.h. Some links in the Userguide will not work, since there targets are only contained in the Referenceguide.
\end{DoxyAttention}


 KESS (KATRIN Electron Scattering in Silicon) is a true event-\/by-\/event simulation of electrons travelling in and interacting with silicon. The basic geometry setup is a two layer silicon bulk material. The first layer is called Deadlayer and the Second Layer is called Detector or Sensitive Volume. KESS only supports electrons and silicon. It is especially made for low energy electrons with kinetic energies of 0-\/50keV. 

\hypertarget{_k_e_s_s_main_KESSContent}{}\subsection{Content}\label{_k_e_s_s_main_KESSContent}

\begin{DoxyEnumerate}
\item \hyperlink{_k_e_s_s_physics}{Physics} 
\item \hyperlink{_k_e_s_s_simulation}{Simulation} 
\item \hyperlink{_k_e_s_s_changelog}{Changelog} 
\item \hyperlink{_k_e_s_s_history}{History} 
\end{DoxyEnumerate}\hypertarget{KESSPhysics}{}\subsection{Physics}\label{KESSPhysics}
The simulation uses the 2 relevant scattering processes through which low energy electrons interact with solid state silicon. Electrons can scatter elastically or inelastically with silicon, other processes like bremsstrahlung canbe neglected. The transferred energy can then induce follow-\/up processes which are also part of KESS. If the energy transfer is bigger than 0 the Silicon atoms will be exited or ionized. Ionized atoms will mainly relax through Auger or Coster-\/Kronig electron emission and is assumed to be 100\% in the simulation. Relaxation through fluorescence is neglected. The processes included so far are: elastic scattering, inelastic scattering, ionization, atomic relaxation and the vacuum-\/silicon transition probabilities.\hypertarget{_k_e_s_s_physics_elastic}{}\paragraph{Elastic Scattering}\label{_k_e_s_s_physics_elastic}
Electrons scattering with the silicon atom potential will change their direction depending on their energy. The energy lost is negligible. The angular change follows a probability density function derived from calculation or experiment. The data tables used in KESS are calculated with the ELSEPA software \mbox{[}1\mbox{]} using relativistic partial wave expansion model and the Dirac-\/Fock atomic potential. Values are included for 1eV to 400keV.

\mbox{[}1\mbox{]} F. Salvat et al. Comp. Phys. Com. 165 (2005) 157190\hypertarget{_k_e_s_s_physics_inelastic}{}\paragraph{Inelastic Scattering}\label{_k_e_s_s_physics_inelastic}
An incoming electron scattering on other electrons bound in the Silicon (K,L1,L2,L3,M-\/shell) will loose up to half its energy and transfer it to a shell electron. The energy loss is highly dependent on the incoming electron energy. The scattering angle is determined by the found energy loss. Energy transfer to the K and L shell electrons will lead to ionization. A secondary electron is created and is optionally added to tracking (see below). KESS is using tables of the mean free path (MFP) and probability density functions (PDF) based on Penn's formalism \mbox{[}1\mbox{]}. The doubly differntial cross section is given by \[\frac{d^2 \lambda_{in}^{-1}}{d(\hbar \omega)dq} = \frac{1}{\pi a_0 E} Im \{ \frac{-1} {\epsilon(q,\omega)}\} \frac{1}{q}\] where $ \lambda_{in}$ is the electrom MFP, $ \hbar \omega$ is the energy loss, $ q$ is the momentum transfer for an electron with kinetic energy $ E $, $ \epsilon(q,\omega)$ is the dielectroc function, $ a_0$ is the Bohr radius and $ Im \{ \frac{-1} {\epsilon(q,\omega)}\} $ is the energy loss function. The energy loss function can be derived by optical dielectric data $ \epsilon(\omega) $ in the plasmon pole approximation. The optical energy loss function $ Im \{ \frac{-1} {\epsilon(\omega)}\} $can be extended to the electron loss function $ Im \{ \frac{-1} {\epsilon(q,\omega)}\} $ by \[ Im \{ \frac{-1} {\epsilon(q,\omega)}\} = \frac{\omega_0}{\omega} Im \{ \frac{-1} {\epsilon(\omega_0)}\} \] with $ \omega_0$ beeing the positive solution of the dispersion relation \[\omega_q^2(q,\omega_p)=\omega_p^2+\frac{1}{3} \nu_f^2(\omega_p)q^2+(\frac{\hbar q^1}{2m})^2 \] The data to calculate the optical energy loss function was compiled by H. Bichsel \mbox{[}2\mbox{]} from experimental and theoretical sources. The Penn model cross sections describes the effects of bulk plasmon excitations, interband transitions and inner-\/shell ionization. The first and second moment of the doubly differential cross section, is the MFP and the stopping power (SP) accordingly.

The incident electron changes direction after the scattering. The scattering angle relative to the former trajectory of the electron is given by the binary collision model to be \[ sin \theta = \sqrt{\frac{\Delta E}{E} } \] whereas the azimuthal angle can be sampled isotropically from $ \phi = 2\pi R$ with a random number $R $.

\mbox{[}1\mbox{]} D. R. Penn, Phys. Rev. B. 35, 482(1987) \mbox{[}2\mbox{]} H. Bichsel Rev. Mod. Phys. 60 (1988) 3\hypertarget{_k_e_s_s_physics_ionization}{}\paragraph{Ionization of Si atoms}\label{_k_e_s_s_physics_ionization}
Incident electrons can ionize or excite the target atom, depending on the energy transfer $\Delta E $. If an energy transfer from the incident electron to a shell electron $\Delta E>E_{L3}$ the atom is ionized. For smaller $\Delta E $, the the M-\/shell (or valance band) is excited. The electron that was knocked out during the ionization can be treated as a secondary electron (delta ray) and can produce further scattering, ionizations, etc. The secondary electron has the energy $ E=\Delta E-E_b$ where $E_b$ is the according binding energy of the shell electron. The user can choose the polar emission angle $ \theta_e $ and the azimuthal angle $ \phi_e $ of the SE to be determined by two different models. One is called \char`\"{}momentum conservation\char`\"{} and is correlated with the incident electron trajectory and the energy transfer by $ sin \theta_e = cos \theta $ where $ \theta $ is the angle of the incident angle after scattering in relation to its relative direction before the scattering. Since it is a binary collision, the scattering can be described on one plane, therefore the azimuthal angle is $ \phi_e =\pi + \phi $, with $\phi $ beeing the scattering angle of the incident electron.

\mbox{[}1\mbox{]} D. T. Cromer, D. Liberman, LASL report LA-\/4403 (1970)

The second model included as an option is spherical symmetry. Since the emission is happening in a solid, the emission angle can be influenced by other electrons, the nucleus or the crystal lattice. Therefore one can assume that the SE emission is not correlated to the incident electron trajectory at all and therefore violating momentum conservation (at least in a classical two-\/body sense). The according angles would then be calculated with two different random numbers $ R_1 $ and $ R_2 $ to be \[ cos \theta_e=1-2R_1, \phi_e =2\pi R_2.\]\hypertarget{_k_e_s_s_physics_relaxation}{}\paragraph{Atomic relaxation of Si}\label{_k_e_s_s_physics_relaxation}
Ionized Si atoms relax through the Auger and the Coster-\/Kronig effect. The ionized shell is filled by outer shell electrons and the surplus energy is transferred to another electron, which is then emitted. This process continues until all inner shells are filled and is thereby creating a cascade of SE. Auger electrons are emitted in spherical symmetry with an energy $ E_{Auger}=E_X-E_Y $ where $ X $ and $ Y $ are $K $, $L_1 $ or $ L_{2.3}$.

\mbox{[}1\mbox{]} G. W. Fraser et al. Nucl. Instr. Meth. A 350, 368(1994)\hypertarget{_k_e_s_s_physics_KESSTransProb}{}\paragraph{Transmission probability}\label{_k_e_s_s_physics_KESSTransProb}
The base potential for the electron changes when entering or exciting the silicon detector. In KESS this is approximated by a potential step with the height $ \chi $ which is known as the electron affinity. It is defined as $ \chi = E_{Vac} - E_{CBM} $, the difference between the vacuum and the conduction band minimum potential. Especially for true secondary electrons ($E<50$eV) exiting the bulk silicon this effect is very important. The literature value is $ \chi = 4.05$ eV. One must note, that surface contamination (oxides etc.) and electric fields inside the silicon strongly change the shape and the height of the conduction band minimum. This can lead to negative electron affinity (NEA), a slope in $ E_{CBM} $ inside SI and to big gradients close to the silicon surface. The true SE backscattering probability is therefore not a good way to validate this kind of simulations. However, the potential step approximation is included for completeness.

\mbox{[}1\mbox{]} J. M. Fernandez-\/Varea et al., Surf. Interf. Anal. 37, 824(2005) \hypertarget{KESSSimulation}{}\subsection{Simulation}\label{KESSSimulation}
KESS uses a true event-\/by-\/event Monte Carlo model. The electrons are tracked step-\/by-\/step through the silicon. Each step ends with an interaction (elastic or inelastic scattering). After each step various exit conditions are checked. Since many random numbers are used and low energy electron tracks in silicon are very erratic, many electrons with the same starting conditions have to be simulated to reach conclusive results. It is advisable to use parallel computing (e.g. Ruby script etc). The \hyperlink{class_kassiopeia_1_1_k_e_s_s_step}{KESSStep} is the smallest simulation unit. It moves a \hyperlink{class_kassiopeia_1_1_k_e_s_s_electron}{KESSElectron}, determines all relevant processes and saves the information (optional). \hyperlink{class_kassiopeia_1_1_k_e_s_s_track}{KESSTrack} is a collection of \hyperlink{class_kassiopeia_1_1_k_e_s_s_step}{KESSStep} objects and saves information of the beginning and end of the track as well as energy deposition and exit condition. The \hyperlink{class_kassiopeia_1_1_k_e_s_s_track}{KESSTrack} is what most user would like to hae for analysis. Each primary submitted to the \hyperlink{class_kassiopeia_1_1_k_e_s_s_run_manager}{KESSRunManager} creates a primary track. Secondary \hyperlink{class_kassiopeia_1_1_k_e_s_s_electron}{KESSElectron} created by primary collisions and subsequent relaxation are also creating KESSTracks. Since everything interesting is happening during a \hyperlink{class_kassiopeia_1_1_k_e_s_s_step}{KESSStep} here is a little description.\hypertarget{_k_e_s_s_simulation_KESSStep}{}\paragraph{A KESSStep}\label{_k_e_s_s_simulation_KESSStep}
The scattering inside the silicon can be described by the Poisson distribution. The probability $ P $ for a scattering after travelling a distance $ S $between is given by \[ P(S) = \lambda_T^{-1}e^{\frac{-S}{\lambda_t}}\]\hypertarget{_k_e_s_s_simulation_structure}{}\paragraph{Code structure}\label{_k_e_s_s_simulation_structure}
KESS is configured by a Configfile. Please see KESSConfig.txt as an example. The \hyperlink{class_kassiopeia_1_1_k_e_s_s_run_manager}{KESSRunManager} steers the whole simulation and is the interface to the user or other simulations. E.g. you can define KESSElectrons and give them to the RunManager. You can also extract information through the RunManager after each run to use in your own code. An example on how to interact with the RunManager is found in ex1.cxx.

The internal structure is divided into the logical parts: Run (\hyperlink{class_kassiopeia_1_1_k_e_s_s_run_manager}{KESSRunManager}), track (\hyperlink{class_kassiopeia_1_1_k_e_s_s_track}{KESSTrack}) and step (\hyperlink{class_kassiopeia_1_1_k_e_s_s_step}{KESSStep}).

\mbox{[}7\mbox{]} Goto et al. -\/ J. Appl. Phys., Vol. 96, No. 8, 2004 \mbox{[}8\mbox{]} D.C. Joy et al. -\/ Scanning, Vol. 17, No. 5, 2006 \hypertarget{KESSChangelog}{}\subsection{ChangeLog}\label{KESSChangelog}
\hypertarget{_k_e_s_s_changelog_v14}{}\paragraph{KESS 1.4 in Kassiopeia 1.00.00 (under development)}\label{_k_e_s_s_changelog_v14}

\begin{DoxyItemize}
\item passed major physics evaluation (DCJoy, Goto, FPD data, ...)
\item updated documentation and linked to \hyperlink{namespace_kassiopeia}{Kassiopeia}
\item QM transmission and angular change now added for entering silicon as well -\/ needs structural improvement though -\/ the first and last step in Si should be somewhat special and should be reflected in the stepinformation
\item \char`\"{}major\char`\"{} (eV effect) bugfixes regarding band structure and energy point of reference
\item Added bandstructure related parameters to KESSConfig.txt
\item Moved stuff from \hyperlink{class_kassiopeia_1_1_k_e_s_s_run_manager}{KESSRunManager} to \hyperlink{class_kassiopeia_1_1_k_e_s_s_queues}{KESSQueues}, \hyperlink{class_kassiopeia_1_1_k_e_s_s_processes}{KESSProcesses}, \hyperlink{class_kassiopeia_1_1_k_e_s_s_parameters}{KESSParameters} and \hyperlink{class_kassiopeia_1_1_k_e_s_s_run_info}{KESSRunInfo}
\item Spherical symmetry and momentum conservation are now options for knock-\/on secondary production
\end{DoxyItemize}\hypertarget{_k_e_s_s_changelog_v13}{}\paragraph{Version 1.3 (never released)}\label{_k_e_s_s_changelog_v13}

\begin{DoxyItemize}
\item changed \char`\"{}Dapor symmetry\char`\"{} to real spherical symmetry
\item added \hyperlink{class_kassiopeia_1_1_k_e_s_s_navigator}{KESSNavigator} and moved the position calculation etc. to that
\item \hyperlink{class_kassiopeia_1_1_k_e_s_s_electron}{KESSElectron} is now friends with \hyperlink{class_kassiopeia_1_1_k_e_s_s_navigator}{KESSNavigator} (blame D. Furse! Not me!)
\item added Transmission probabilities to \hyperlink{class_kassiopeia_1_1_k_e_s_s_navigator}{KESSNavigator}
\item added PENN CCS
\item extended ELSEPA down to 3eV
\item added abstract class \hyperlink{_k_e_s_s_i_c_s_8h}{KESSICS.h} to handle Ionization probabilities.
\item added KESSPhotoAbsorption process to \hyperlink{_k_e_s_s_i_c_s_8h}{KESSICS.h}
\item removed KESSCrosssections. All CS are registered at \hyperlink{class_kassiopeia_1_1_k_e_s_s_step}{KESSStep} now.
\item backscattering performance tested. PENN, ELSEPA, Relaxation and PhotoAbsorption work fine!
\item added bandGap energy for Si to KESSConfig.txt. Use 0.0 until there is an energy deposition model.
\item added option to just save backscattered electrons in the trackTree to KESSConfig.txt to save harddisk space.
\item KESS is now KATRIN coding standard compliant (sort of)
\end{DoxyItemize}\hypertarget{_k_e_s_s_changelog_v12}{}\paragraph{Version 1.2 (2009-\/10-\/21)}\label{_k_e_s_s_changelog_v12}

\begin{DoxyItemize}
\item cleaned rRunTree
\item made KESSCrossSections.h the wrapper class for all cross sections
\item \hyperlink{_k_e_s_s_e_c_s_8h}{KESSECS.h} and \hyperlink{_k_e_s_s_c_c_s_8h}{KESSCCS.h} are abstract classes for the according cross sections
\item \hyperlink{_k_e_s_s_e_c_s_elsepa_8h}{KESSECSElsepa.h} and \hyperlink{_k_e_s_s_c_c_s_bethe_fano_8h}{KESSCCSBetheFano.h} are the concrete implementations
\item Added option to choose Processes/cross sections to the KESSConfig.txt
\item cleaned StepTree
\item \hyperlink{_k_e_s_s_electron_queue_8h}{KESSElectronQueue.h} can be cleared now
\end{DoxyItemize}\hypertarget{_k_e_s_s_changelog_v11}{}\paragraph{Version 1.1 (2009-\/09-\/02)}\label{_k_e_s_s_changelog_v11}

\begin{DoxyItemize}
\item Changed table lookup routines to STL standard -\/$>$ one run now takes 7\% of the time it used to!!
\end{DoxyItemize}\hypertarget{_k_e_s_s_changelog_v10}{}\paragraph{Version 1.0 (2009-\/08-\/27)}\label{_k_e_s_s_changelog_v10}

\begin{DoxyItemize}
\item Added \hyperlink{class_kassiopeia_1_1_k_e_s_s_primary_track_info}{KESSPrimaryTrackInfo} and KESSRunManager::getLastPrimaryTrackInfo() for implementation in KDES
\item Removed {\itshape SIFermiEnergy\/} -\/$>$ only usage was track killing at LowestEnergyForTracking -\/ SiFermE
\item {\itshape fGlobalElectronID\/} and {\itshape fGlobalTrackID\/} only accessible by KESSRunManager::getNextGlobalTrackID() and KESSRunManager::getNextGlobalElectronID() -\/-\/$>$ controlled by \hyperlink{class_kassiopeia_1_1_k_e_s_s_run_manager}{KESSRunManager} now!
\item Cleaned up \hyperlink{class_kassiopeia_1_1_k_e_s_s_run_manager}{KESSRunManager}
\item Added {\itshape time\/} (absolute time) and {\itshape creationTime\/} to \hyperlink{class_kassiopeia_1_1_k_e_s_s_electron}{KESSElectron}
\item Added \hyperlink{class_kassiopeia_1_1_k_e_s_s_track_info}{KESSTrackInfo}
\item Cleaned up \hyperlink{class_kassiopeia_1_1_k_e_s_s_output}{KESSOutput}
\item \hyperlink{class_kassiopeia_1_1_k_e_s_s_track_info}{KESSTrackInfo} and \hyperlink{class_kassiopeia_1_1_k_e_s_s_primary_track_info}{KESSPrimaryTrackInfo} are now objects that are written to \hyperlink{class_kassiopeia_1_1_k_e_s_s_output}{KESSOutput}. They can also be saved in a queue, if this is required at some point.
\item New option in KESSConfig {\itshape CreatePrimaryTrackTree:1\/} 
\item KESSElectron::move also takes care of timing.
\item You can assign an {\itshape externalID\/} to a \hyperlink{class_kassiopeia_1_1_k_e_s_s_electron}{KESSElectron} which isn't touched by the KESS
\item Check for configFile will now ask for input if it failed
\item a precompiled lib is in ./lib
\item {\itshape kess.cxx\/} and {\itshape KESSConfig.txt\/} moved to examples/
\item {\itshape kess.cxx\/} renamed to {\itshape ex1.cxx\/} 
\item examples now work with {\itshape libKESS.so\/} 
\item Installation section added
\end{DoxyItemize}\hypertarget{_k_e_s_s_changelog_v003}{}\paragraph{Version 0.0.3}\label{_k_e_s_s_changelog_v003}

\begin{DoxyItemize}
\item Added \hyperlink{class_kassiopeia_1_1_k_e_s_s_conf_reader}{KESSConfReader}. You can now use a config file instead of the \hyperlink{class_kassiopeia_1_1_k_e_s_s_run_manager}{KESSRunManager} setters.
\item Added Hack for Martin \char`\"{}Quick'n'Dirty\char`\"{} Babutzka to get energy deposit of last track KESSRunManager::getEDepDetectorOfLastTrack.
\item Now with Uniform exit conditions in \hyperlink{class_kassiopeia_1_1_k_e_s_s_electron}{KESSElectron}
\item Validation of \hyperlink{class_kassiopeia_1_1_k_e_s_s_conf_reader}{KESSConfReader} readin (do the parameters exist?)
\item Added KESSRunManager::setVerbose() 0=none 10=all 
\end{DoxyItemize}\hypertarget{KESSHistory}{}\subsection{History}\label{KESSHistory}
\hypertarget{_k_e_s_s_history_history}{}\paragraph{History}\label{_k_e_s_s_history_history}
The code was originally written in Fortran by Zine-\/El-\/Abidine Chaoui (Physics department, Faculty of Science, University of Setif, Algeria, \href{mailto:z_chaoui@yahoo.fr}{\tt z\_\-chaoui@yahoo.fr}) from 2005 -\/ 2007. It included:
\begin{DoxyItemize}
\item Dirac Partial Wave cross sections for elastic cross sections and Bethe-\/Fano/Penn dielectric formalism for the inelastic cross sections
\item The Bethe-\/Fano inelastic cross sections were calculated by Hans Bichsel (CENPA, University of Washington, Seattle, USA, \href{mailto:bichsel@npl.washington.edu}{\tt bichsel@npl.washington.edu})
\end{DoxyItemize}

Pascal Renschler (Institute for Experimental Nuclear Physics, Karlsruhe Institute of Technology, Germany, \href{mailto:pascal.renschler@gmx.de}{\tt pascal.renschler@gmx.de}) translated the code to c++ starting from 2008:
\begin{DoxyItemize}
\item fully object oriented code structure
\item own electron object
\item \hyperlink{namespace_kassiopeia_1_1_r_o_o_t}{ROOT} random generator (TRandom3.h)
\item full \hyperlink{namespace_kassiopeia_1_1_r_o_o_t}{ROOT} output
\item all processes included as options in the \hyperlink{class_kassiopeia_1_1_k_e_s_s_run_manager}{KESSRunManager}
\item the code is easily parallelized through e.g. a small Ruby script
\end{DoxyItemize}

From 2008 to 2010 Z. Chaoui and P. Renschler included the following:
\begin{DoxyItemize}
\item Ionization process (Fermi Virtual Photon/Photo Absorption Ionization relative probabilities)
\item Relaxation Process (Auger-\/ and Coster-\/Kroning-\/electron emission), fluorescence is excluded
\item Surface escape process
\item Knock-\/on secondary electron emission (spherical symmetry or binary collision model)
\item Secondary cascade
\end{DoxyItemize}

In march 2010 Z. Chaoui and P. Renschler did the following:
\begin{DoxyItemize}
\item extended the elastic cross sections down to 0.1eV
\item major physics evaluation
\end{DoxyItemize}

In September 2010 KESS became a module of \hyperlink{namespace_kassiopeia}{Kassiopeia} (the KATRIN Simulation framework). 



